%
% $Id: SANDExampleReportNotstrict.tex,v 1.26 2009-05-01 20:59:19 rolf Exp $
%
% This is an example LaTeX file which uses the SANDreport class file.
% It shows how a SAND report should be formatted, what sections and
% elements it should contain, and how to use the SANDreport class.
% It uses the LaTeX report class, but not the strict option.
%
% Get the latest version of the class file and more at
%    http://www.cs.sandia.gov/~rolf/SANDreport
%
% This file and the SANDreport.cls file are based on information
% contained in "Guide to Preparing {SAND} Reports", Sand98-0730, edited
% by Tamara K. Locke, and the newer "Guide to Preparing SAND Reports and
% Other Communication Products", SAND2002-2068P.
% Please send corrections and suggestions for improvements to
% Rolf Riesen, Org. 9223, MS 1110, rolf@cs.sandia.gov
%
\documentclass[pdf,12pt,report]{SANDreport}
\usepackage{algpseudocode}
\usepackage{amsthm}
\usepackage{booktabs}
\usepackage{calc}
\usepackage{eso-pic}
\usepackage{fancyhdr}
\usepackage{ifthen}
\usepackage{indentfirst}
\usepackage{geometry}
\usepackage{graphicx}
\usepackage[colorlinks, bookmarksopen, %pagebackref=true, backref=page,
             linkcolor={blue},
             anchorcolor={black},
             citecolor={blue},
             filecolor={magenta},
             menucolor={blue},
             pagecolor={red},
             plainpages=false,pdfpagelabels,
             pdfauthor={Andrey Prokopenko, Chris Siefert, Jonathan J. Hu, Mark
             Hoemmen, Alicia Klinvex},
             pdftitle={Ifpack2 User's Guide},
             pdfkeywords={Ifpack2,preconditioners,guide,user},
             urlcolor={blue}]{hyperref}
\usepackage{listings}
\usepackage{mathptmx}	% Use the Postscript Times font
\usepackage{multirow}
\usepackage{pifont}
\usepackage[FIGBOTCAP,normal,bf,tight]{subfigure}
\usepackage{tabularx}
\usepackage{verbatim}
\usepackage{xspace}
\usepackage{flowchart} % also loads tikz
\usepackage{algorithm}
\usetikzlibrary{arrows}

%\usepackage{draftwatermark}
%\SetWatermarkScale{.5}

\algrenewcommand{\algorithmiccomment}[1]{\hskip3em // #1}

%%%%%%%%%%%%%%%%%%%%%%%%%%%%%%%%%%%%%%%%%%%%%%%%%%%%%%%%%%%%%%%%%%%%%%%%%%%%%%%%%%%%%%%%%%%%%%%%%%%%%%%%%%%%%%%%%%%%%
% Want larger todonotes on margins?
% First, use package showframes to show the frames
% Then, adjust the geometry
% NOTE: this must be removed in the final version
% \usepackage{showframe}
% \setlength{\marginparwidth}{3.5cm}

% Add disable to todonotes options to disable all TODO notes without removing them
% \usepackage[colorinlistoftodos,prependcaption,textsize=small]{todonotes}

% \usepackage{xargs}
% \usepackage{soul}
% \newcommandx{\fix}     [3][1=]{\todo[linecolor=red,backgroundcolor=red!25,bordercolor=red,#1]{\textbf{#2: }#3}}
% \newcommandx{\unsure}  [3][1=]{\todo[linecolor=green,backgroundcolor=green!25,bordercolor=green,#1]{\textbf{#2: }#3}}
% \newcommandx{\improve} [3][1=]{\todo[linecolor=blue,backgroundcolor=blue!25,bordercolor=blue,#1]{\textbf{#2: }#3}}
% \newcommandx{\info}    [3][1=]{\todo[linecolor=gray,backgroundcolor=gray!25,bordercolor=gray,#1]{\textbf{#2: }#3}}
% \newcommandx{\fixhl}   [2]    {\texthl{#1}\fix{#2}}
%%%%%%%%%%%%%%%%%%%%%%%%%%%%%%%%%%%%%%%%%%%%%%%%%%%%%%%%%%%%%%%%%%%%%%%%%%%%%%%%%%%%%%%%%%%%%%%%%%%%%%%%%%%%%%%%%%%%%


% If you want to relax some of the SAND98-0730 requirements, use the "relax"
% option. It adds spaces and boldface in the table of contents, and does not
% force the page layout sizes.
% e.g. \documentclass[relax,12pt]{SANDreport}
%
% You can also use the "strict" option, which applies even more of the
% SAND98-0730 guidelines. It gets rid of section numbers which are often
% useful; e.g. \documentclass[strict]{SANDreport}



% ---------------------------------------------------------------------------- %
%
% Set the title, author, and date
%
\title{Ifpack2 User's Guide 1.0 \\
(Trilinos version 12.6)}

\author{
  Andrey Prokopenko \\
  Scalable Algorithms \\
  Sandia National Laboratories\\
  Mailstop 1318 \\
  P.O.~Box 5800 \\
  Albuquerque, NM 87185-1318\\
  aprokop@sandia.gov\\
  \and
  Christopher Siefert \\
  Computational Math \& Algorithms \\
  Sandia National Laboratories\\
  Mailstop 1318 \\
  P.O.~Box 5800 \\
  Albuquerque, NM 87185-1318 \\
  \and
  Jonathan J. Hu \\
  Scalable Algorithms \\
  Sandia National Laboratories\\
  Mailstop 9159 \\
  P.O.~Box 0969 \\
  Livermore, CA 94551-0969\\
  jhu@sandia.gov \\
  \and
  Mark Hoemmen \\
  Scalable Algorithms \\
  Sandia National Laboratories\\
  Mailstop 1320 \\
  P.O.~Box 5800 \\
  Albuquerque, NM 87185-1318\\
  mhoemme@sandia.gov\\
  \and
  Alicia Klinvex \\
  Scalable Algorithms \\
  Sandia National Laboratories\\
  Mailstop 1320 \\
  P.O.~Box 5800 \\
  Albuquerque, NM 87185-1318\\
  amklinv@sandia.gov\\
}

% There is a "Printed" date on the title page of a SAND report, so
% the generic \date should generally be empty.
\date{}

\def\optionbox#1#2{\noindent$\hphantom{ii}${\parbox[t]{1.5in}{\it
#1}}{\parbox[t]{4.8in}{#2}} \\[1.1em]}

\def\choicebox#1#2{\noindent$\hphantom{th}$\parbox[t]{2.5in}{\sf
#1}\qquad\parbox[t]{3.55in}{#2}\\[0.8em]}

\def\structbox#1#2{\noindent$\hphantom{hix}${\parbox[t]{2.10in}{\it
#1}}{\parbox[t]{3.9in}{#2}} \\[.02cm]}

\def\protobox#1{\vspace{2em}{\flushleft{\bf Prototype}
\hrulefill}\flushleft{\fbox{\parbox[t]{6in}{\vspace{1em}{\sf
#1}\vspace{1em}}}}}


%\setlength{\oddsidemargin} {0.1\oddsidemargin}
%\setlength{\evensidemargin}{0.5\evensidemargin}
%\setlength{\topmargin}     {0.0\topmargin}
%\setlength{\textheight}    {1.16\textheight}
%\setlength{\textwidth}     {1.35\textwidth}

\newcommand{\amesos}       {\textsc{Amesos}\xspace}
\newcommand{\amesostwo}    {\textsc{Amesos2}\xspace}
\newcommand{\anasazi}      {\textsc{Anasazi}\xspace}
\newcommand{\aztecoo}      {\textsc{AztecOO}\xspace}
\newcommand{\belos}        {\textsc{Belos}\xspace}
\newcommand{\epetra}       {\textsc{Epetra}\xspace}
\newcommand{\epetraext}    {\textsc{EpetraExt}\xspace}
\newcommand{\galeri}       {\textsc{Galeri}\xspace}
\newcommand{\ifpack}       {\textsc{Ifpack}\xspace}
\newcommand{\ifpacktwo}    {\textsc{Ifpack2}\xspace}
\newcommand{\kokkosclassic}{\textsc{KokkosClassic}\xspace}
\newcommand{\loca}         {\textsc{Loca}\xspace}
\newcommand{\ml}           {\textsc{ML}\xspace}
\newcommand{\muelu}        {\textsc{MueLu}\xspace}
\newcommand{\nox}          {\textsc{NOX}\xspace}
\newcommand{\stratimikos}  {\textsc{Stratimikos}\xspace}
\newcommand{\teuchos}      {\textsc{Teuchos}\xspace}
\newcommand{\tpetra}       {\textsc{Tpetra}\xspace}
\newcommand{\tpetrakernels}{\textsc{TpetraKernels}\xspace}
\newcommand{\trilinos}     {\textsc{Trilinos}\xspace}
\newcommand{\xpetra}       {\textsc{Xpetra}\xspace}
\newcommand{\zoltan}       {\textsc{Zoltan}\xspace}
\newcommand{\zoltantwo}    {\textsc{Zoltan2}\xspace}

\newcommand{\klu}          {\textsc{Klu}\xspace}
\newcommand{\metis}        {\textsc{Metis}\xspace}
\newcommand{\mumps}        {\textsc{Mumps}\xspace}
\newcommand{\umfpack}      {\textsc{Umfpack}\xspace}
\newcommand{\superlu}      {\textsc{SuperLU}\xspace}
\newcommand{\superludist}  {\textsc{SuperLU\_dist}\xspace}
\newcommand{\parmetis}     {\textsc{ParMetis}\xspace}
\newcommand{\paraview}     {\textsc{ParaView}\xspace}

\newcommand{\parameterlist}{\texttt{Teuchos::ParameterList}\xspace}

\newcommand \trilinosWeb   {trilinos.sandia.gov\xspace}

\newcommand \true {\texttt{true}}
\newcommand \false{\texttt{false}}

\newcommand{\be}  {\begin{enumerate}}
\newcommand{\ee}  {\end{enumerate}}

\newcommand{\comm}[2]{\bigskip
                      \begin{tabular}{|p{4.5in}|}\hline
                      \multicolumn{1}{|c|}{{\bf Comment by #1}}\\ \hline
                      #2\\ \hline
                      \end{tabular}\\
                      \bigskip
                     }

% specify a Teuchos::Parameter.
%  #1 = parameter name
%  #2 = type
%  #3 = description
%  #4 = default value
% example:  \ccc{fact: ilut level-of-fill}{int}{1}{Sets the level-of-fill for ILUT.}
\newcommand{\ccc}[4]{\choicebox{\tt "#1"}{[{\tt #2}] #4 {\bf Default:~}#3.}}
\newcommand{\cccc}[2]{\choicebox{\tt "#1"}{#2}}


\newtheorem*{mycomment}{\ding{42}}
\newtheoremstyle{plain}
  {\topsep}   % ABOVESPACE
  {\topsep}   % BELOWSPACE
  {\normalfont}  % BODYFONT
  {0pt}       % INDENT (empty value is the same as 0pt)
  {\bfseries} % HEADFONT
  {}         % HEADPUNCT
  {5pt plus 1pt minus 1pt} % HEADSPACE
  {}          % CUSTOM-HEAD-SPEC

% further declarations and additional commands
\definecolor{hellgelb}{rgb}{1,1,0.8}   % background color for C++ listings
\definecolor{darkgreen}{rgb}{0.0, 0.2, 0.13}
%\definecolor{hellrot}{HTML}{FFA4C2}    % background color for xml files

% settings for listings package
\lstset{
  backgroundcolor=\color{hellgelb},
  basicstyle=\ttfamily\small,
  breakautoindent=true,
  breaklines=true,
  captionpos=b,
  columns=flexible,
  commentstyle=\color{darkgreen},
  extendedchars=true,
  float=hbp,
  frame=single,
  identifierstyle=\color{black},
  keywordstyle=\color{blue},
  numbers=none,
  numberstyle=\tiny,
  showspaces=false,
  showstringspaces=false,
  stringstyle=\color{purple},
  tabsize=2,
}


% ---------------------------------------------------------------------------- %
% Set some things we need for SAND reports. These are mandatory
%
\SANDnum{SAND2016-5338}
\SANDprintDate{June 2016}
\SANDauthor{Andrey Prokopenko, Christopher M. Siefert, Jonathan J. Hu, \\Mark
Hoemmen, Alicia Klinvex}


% ---------------------------------------------------------------------------- %
% Include the markings required for your SAND report. The default is "Unlimited
% Release". You may have to edit the file included here, or create your own
% (see the examples provided).
%
% \include{MarkUR} % Not needed for unlimted release reports


% ---------------------------------------------------------------------------- %
% The following definition does not have a default value and will not
% print anything, if not defined
%
%\SANDsupersed{SAND1901-0001}{January 1901}
%\input{MarkOUO}


% ---------------------------------------------------------------------------- %
%
% Start the document
%
\begin{document}

    \maketitle

    % ------------------------------------------------------------------------ %
    % An Abstract is required for SAND reports
    %
    \begin{abstract}
	%This is the definitive user guide for the \muelu{} library in Trilinos version XX.YY.
%\muelu{} is a C++ multigrid framework that can work with either the \epetra or \tpetra linear
%algebra libraries.
%\muelu{} provides a variety of aggregation-based multigrid algorithms,
%including smoothed aggregation algebraic multigrid (AMG), Petrov-Galerkin AMG, and AMG for
%Maxwell's equations, as well as many different types of smoothers.
%\muelu{} is templated on the index, scalar, and compute node types.
%Thus it is possible to use \muelu{} on problems with scalar types other than double, on very
%large problems, and to exploit node-level parallelism.

This is the official user guide for \muelu{} multigrid library in Trilinos
version~\input{version}. This guide provides an overview of \muelu, its capabilities, and
instructions for new users who want to start using \muelu{} with a minimum of
effort. Detailed information is given on how to drive \muelu{} through its XML
interface. Links to more advanced use cases are given. This guide gives
information on how to achieve good parallel performance, as well as how to
introduce new algorithms. Finally, readers will find a comprehensive listing of
available \muelu{} options.  {\em Any options not documented in this manual
should be considered strictly experimental.}

    \end{abstract}


    % ------------------------------------------------------------------------ %
    % An Acknowledgement section is optional but important, if someone made
    % contributions or helped beyond the normal part of a work assignment.
    % Use \section* since we don't want it in the table of context
    %
    \clearpage
    \chapter*{Acknowledgment}
	We would like to thank Brian Barrett and Doug Doerfler for access to the Niagara and Clovertown architectures,
respectively.  We also would like to thank Danny Dunlavy and Chris Siefert for providing us with the test matrices.
In addition, we would like to thank Mike Heroux, Chris Siefert and Jim Willenbring for reviewing the interface design
and answering questions along the way.  Jim's partial OSKI implimentation of an interface within Kokkos helped serve as
a model for our development.  Finally, we would also like to thank Rich Vuduc for his help with Oski-related questions.


    % ------------------------------------------------------------------------ %
    % The table of contents and list of figures and tables
    % Comment out \listoffigures and \listoftables if there are no
    % figures or tables. Make sure this starts on an odd numbered page
    %
    \cleardoublepage		% TOC needs to start on an odd page
    \tableofcontents
    \listoffigures
    \listoftables


    % ---------------------------------------------------------------------- %
    % An optional preface or Foreword
    %\clearpage
    %\chapter*{Preface}
    %\addcontentsline{toc}{chapter}{Preface}
	%\input{CommonPreface}


    % ---------------------------------------------------------------------- %
    % An optional executive summary
    %\clearpage
    %\chapter*{Summary}
    %\addcontentsline{toc}{chapter}{Summary}
	%\input{CommonSummary}


    % ---------------------------------------------------------------------- %
    % An optional glossary. We don't want it to be numbered
    %\clearpage
    %\chapter*{Nomenclature}
    %\addcontentsline{toc}{chapter}{Nomenclature}
    %\begin{description}
	%\item[dry spell]
	%    using a dry erase marker to spell words
	%\item[dry wall]
	%    the writing on the wall
	%\item[dry humor]
	%    when people just do not understand
	%\item[DRY]
	%    Don't Repeat Yourself
    %\end{description}


    % ---------------------------------------------------------------------- %
    % This is where the body of the report begins; usually with an Introduction
    %
    \SANDmain		% Start the main part of the report

    %-----------------------------%
    % \chapter{Introduction}\label{sec:introduction}
    %-----------------------------%
    % %
% What this document is, what prior knowledge is assumed, and
% brief outline of what is contained in it.
% 
% PLEASE FEEL FREE TO RE-WRITE ANY OF THIS!!
%
\chapter{Introduction}

This document provides an introduction to data partitioning
and load balancing using the Zoltan Toolkit (version 3.1).  
Zoltan is a parallel toolkit for load balancing
(and several other combinatorial scientific computing tasks) 
targeted to the high performance computing community.  
This tutorial only covers load balancing,
and other capabilities in Zoltan are barely mentioned.
The Zoltan library is written in C but can be
linked with C, C++ and Fortran90 applications.  
It requires MPI to run in parallel.
Most of the examples in this tutorial are C language examples.

In Chapter~\ref{cha:lb} we
discuss the load balancing problem in general and describe
common algorithms, which are also supported in Zoltan.
Zoltan's role in addressing the load balancing problem is to 
efficiently partition,
or repartition, the problem data across mulitple processes while an
application is running, when requested to do so by the application.

% In Chapter~\ref{cha:partitioning} we
% describe in more detail the partitioning
% methods available in Zoltan, and try provide a guide
% to choosing the method or methods best suited to your problem.

In Chapter~\ref{cha:using} we briefly explain the interface
through which your application employs Zoltan.  This topic is
convered in more detail in the
Zoltan User's Guide available at
\url{http://www.cs.sandia.gov/Zoltan/ug_html/ug.html}.
An example can be worth a thousand words, so you may wish to 
skip this chapter and look through the examples in the last section first.

The final chapter of this document (~\ref{cha:ex}) provides
source code examples for simple applications that use Zoltan.
These examples may be found in the \textbf{examples} directory
of the Zoltan source code.  Zoltan is open source and freely available.

You will find additional documents and publications at the
Zoltan web site at \url{http://www.cs.sandia.gov/Zoltan/}.
The User's Guide is a very useful reference and provides
many details we could not cover here.



    %-----------------------------%
    \chapter{Getting Started}\label{sec:getting started}
    This section is meant to get you using \ifpacktwo{} as quickly as possible.
\S\ref{sec:overview} gives a brief overview of \ifpacktwo{}.
\S\ref{sec:configuration_and_build} lists \ifpacktwo{}'s dependencies on other
\trilinos{} libraries and provides a sample cmake configuration line. Finally,
some examples of code are given in~\S\ref{sec:examples in code}.

\section{Overview of \ifpacktwo{}}
\label{sec:overview}
\ifpacktwo{} is a C++ linear solver library in the \trilinos{} project~\cite{Heroux2012}.
It originally began as a migration of \ifpack{} package capabilities to a new linear
algebra stack. While it retains some commonalities with the original package, it
has since diverged significantly from it and should be treated as completely
independent package.

\ifpacktwo{} only works with \tpetra{}~\cite{TpetraURL} matrix,
vector, and map types. Like \tpetra{}, it allows for different ordinal
(index) and scalar types. \ifpacktwo{} was designed to be efficient on a wide
range of computer architectures, from workstations to supercomputers~\cite{Lin2014}.
It relies on the ``MPI+X" principle, where ``X'' can be threading or
CUDA\@. The ``X'' portion, node-level parallelism, is controlled by a node
template type. Users should refer to \tpetra{}'s documentation for information
about node and device types.

\ifpacktwo provides a number of different solvers, including
\begin{itemize}
  \item Jacobi, Gauss-Seidel, polynomial, distributed relaxation;
  \item domain decomposition solvers;
  \item incomplete factorizations.
\end{itemize}
This list of solvers is not exhaustive. Instead, references for further
information are provided throughout the text. There are many excellent
references for iterative methods, including~\cite{Saad2003}.

Complete information on available capabilities and options can be found
in~\S\ref{sec:options}.

\section{Configuration and Build}\label{sec:configuration_and_build}

\ifpacktwo{} requires a C++11 compatible compiler for compilation. The
minimum required version of compilers are GCC (4.7.2 and later),
Intel (13 and later), and clang (3.5 and later).

\subsection{Dependencies}

Table~\ref{tab:dependencies} enumerates the dependencies of \ifpacktwo. Certain
dependencies are optional, whereas others are required.  Furthermore,
\ifpacktwo's tests depend on certain libraries that are not required if you only
want to link against the \ifpacktwo library and do not want to compile its
tests. Additionally, some functionality in \ifpacktwo{} may depend on other
Trilinos packages (for instance, \amesostwo{}) that may require additional
dependencies. We refer to the documentation of those packages for a full list of
dependencies.

\begin{table}[ht]
  \centering
  \begin{tabular}{p{3.5cm} c c c c}
    \toprule
    \multirow{2}{*}{Dependency} & \multicolumn{2}{c}{Library} & \multicolumn{2}{c}{Testing} \\
    \cmidrule(r){2-3} \cmidrule(l){4-5} & Required & Optional & Required & Optional  \\
    \midrule
    % \belos                       & $\times$ &          & $\times$ & \\
    \teuchos                     & $\times$ &          & $\times$ & \\
    \tpetra                      & $\times$ &          & $\times$ & \\
    \tpetrakernels               & $\times$ &          &          & \\
    \amesostwo                   &          & $\times$ &          & $\times$  \\
    \galeri                      &          &          &          & $\times$  \\
    \xpetra                      &          & $\times$ &          & $\times$  \\
    \zoltantwo                   &          & $\times$ &          & $\times$  \\
    \textsc{ThyraTpetraAdapters} &          & $\times$ &          & \\
    \textsc{ShyLUBasker}         &          & $\times$ &          & $\times$ \\
    \textsc{ShyLUHTS}            &          & $\times$ &          & $\times$ \\
    \midrule
    % BLAS                         & $\times$ &          & $\times$ & \\
    % LAPACK                       & $\times$ &          & $\times$ & \\
    MPI                          &          & $\times$ &          & $\times$  \\
    % Cholmod                      &          & $\times$ &          & $\times$  \\
    % SuperLU 4.3                  &          & $\times$ &          & $\times$  \\
    % QD                           &          & $\times$ &          & $\times$  \\
    \bottomrule
  \end{tabular}
  \caption{\label{tab:dependencies}\ifpacktwo{}'s required and optional dependencies,
    subdivided by whether a dependency is that of the \ifpacktwo{}{} library itself
    (\textit{Library}), or of some \ifpacktwo{}{} test (\textit{Testing}). }
\end{table}

\amesostwo and \superlu are necessary if you want to use either a sparse direct
solve or ILUTP as a subdomain solve in processor-based domain decomposition.
\zoltantwo and \xpetra are necessary if you want to reorder a matrix (e.g.,
reverse Cuthill McKee).

\subsection{Configuration}
The preferred way to configure and build \ifpacktwo{} is to do that outside of the source directory.
Here we provide a sample configure script that will enable \ifpacktwo{} and all of its optional dependencies:
\begin{lstlisting}
  export TRILINOS_HOME=/path/to/your/Trilinos/source/directory
  cmake -D BUILD_SHARED_LIBS:BOOL=ON \
        -D CMAKE_BUILD_TYPE:STRING="RELEASE" \
        -D CMAKE_CXX_FLAGS:STRING="-g" \
        -D Trilinos_ENABLE_EXPLICIT_INSTANTIATION:BOOL=ON \
        -D Trilinos_ENABLE_TESTS:BOOL=OFF \
        -D Trilinos_ENABLE_EXAMPLES:BOOL=OFF \
        -D Trilinos_ENABLE_Ifpack2:BOOL=ON \
        -D Ifpack2_ENABLE_TESTS:STRING=ON \
        -D Ifpack2_ENABLE_EXAMPLES:STRING=ON \
        -D TPL_ENABLE_BLAS:BOOL=ON \
        -D TPL_ENABLE_MPI:BOOL=ON \
        ${TRILINOS_HOME}
\end{lstlisting}

\noindent
More configure examples can be found in \texttt{Trilinos/sampleScripts}.
For more information on configuring, see the \trilinos Cmake Quickstart guide \cite{TrilinosCmakeQuickStart}.

\section{Interface to \ifpacktwo{} methods}
All \ifpacktwo operators inherit from the base class
\texttt{Ifpack2::Preconditioner}. This in turn inherits from
\texttt{Tpetra::Operator}. Thus, you may use an \ifpacktwo operator anywhere
that a \texttt{Tpetra::Operator} works. For example, you may use \ifpacktwo operators
directly as preconditioners in \trilinos' \belos package of iterative solvers.

You may either create an \ifpacktwo operator directly, by using the class and
options that you want, or by using \texttt{Ifpack2::Factory}. Some of
\ifpacktwo preconditioners only accept a \texttt{Tpetra::\\CrsMatrix} instance as
input, while others also may accept a \texttt{Tpetra::RowMatrix} (the base class
of \texttt{Tpetra::CrsMatrix}). They will decide at run time whether the input
\texttt{Tpetra::RowMatrix} is an instance of the right subclass.

\texttt{Ifpack2::Preconditioner} includes the following methods:
\begin{itemize}
  \item \texttt{initialize()}

    Performs all operations based on the graph of the matrix (without
    considering the numerical values).

  \item \texttt{compute()}

    Computes everything required to apply the preconditioner, using the matrix's
    values.

  \item \texttt{apply()}

    Applies or ``solves with'' the preconditioner.
\end{itemize}
Every time that \texttt{initialize()} is called, the object destroys all the
previously allocated information, and reinitializes the preconditioner. Every
time \texttt{compute()} is called, the object recomputes the actual values of the
preconditioner.

An \ifpacktwo preconditioner may also inherit from
\texttt{Ifpack2::CanChangeMatrix} class in order to express that users can
change its matrix (the matrix that it preconditions) after construction using a
\texttt{setMatrix} method.  Changing the matrix puts the preconditioner back in
an ``pre-initialized'' state.  You must first call \texttt{initialize()}, then
\texttt{compute()}, before you may call \texttt{apply()} on this preconditioner.
Depending on the implementation, it may be legal to set the matrix to null. In
that case, you may not call \texttt{initialize()} or \texttt{compute()} until
you have subsequently set a nonnull matrix.

\textbf{Warning.} If you are familiar with the \ifpack package~\cite{ifpack}, please be aware
that the behaviour of the \ifpacktwo preconditioner is different from \ifpack.
In \ifpack, the \texttt{ApplyInverse()} method applies or ``solves with'' the
preconditioner $M^{-1}$, and the \texttt{Apply()} method ``applies'' the
preconditioner $M$. In \ifpacktwo, the \texttt{apply()} method applies or
``solves with'' the preconditioner $M^{-1}$. \ifpacktwo has no method comparable
to \ifpack's \texttt{Apply()}.

\section{Example: \ifpacktwo preconditioner within \belos}\label{sec:examples in code}

The most commonly used scenario involving \ifpacktwo{} is using one of its
preconditioners preconditioners inside an iterative linear solver. In
\trilinos{}, the \belos{} package provides important Krylov subspace methods (such
as preconditioned CG and GMRES).

At this point, we assume that the reader is comfortable with \teuchos{} referenced-counted
pointers (RCPs) for memory management (an introduction to RCPs can be found
in~\cite{RCP2010}) and the \parameterlist class~\cite{TeuchosURL}.

First, we create an \ifpacktwo{} preconditioner using a provided \parameterlist
\begin{lstlisting}[language=C++]
 typedef Tpetra::CrsMatrix<Scalar, LocalOrdinal, GlobalOrdinal, Node>
   Tpetra_Operator;

 Teuchos::RCP<Tpetra_Operator> A;
 // create A here ...
 Teuchos::ParameterList paramList;
 paramList.set( "chebyshev: degree", 1 );
 paramList.set( "chebyshev: min eigenvalue", 0.5 );
 paramList.set( "chebyshev: max eigenvalue", 2.0 );
 // ...
 Ifpack2::Factory factory;
 RCP<Ifpack2::Ifpack2Preconditioner<> > ifpack2Preconditioner;
 ifpack2Preconditioner = factory.create( "CHEBYSHEV", A )
 ifpack2Preconditioner->setParameters( paramList );
 ifpack2Preconditioner->initialize();
 ifpack2Preconditioner->compute();
\end{lstlisting}

Besides the linear operator $A$, we also need an initial guess vector for the
solution $X$ and a right hand side vector $B$ for solving a linear system.
\begin{lstlisting}[language=C++]
 typedef Tpetra::Map<LocalOrdinal, GlobalOrdinal, Node> Tpetra_Map;
 typedef Tpetra::MultiVector<Scalar, LocalOrdinal, GlobalOrdinal, Node>
   Tpetra_MultiVector;

 Teuchos::RCP<const Tpetra_Map> map = A->getDomainMap();

 // create initial vector
 Teuchos::RCP<Tpetra_MultiVector> X =
   Teuchos::rcp( new Tpetra_MultiVector(map, numrhs) );

 // create right-hand side
 X->randomize();
 Teuchos::RCP<Tpetra_MultiVector> B =
   Teuchos::rcp( new Tpetra_MultiVector(map, numrhs) );
 A->apply( *X, *B );
 X->putScalar( 0.0 );
\end{lstlisting}
To generate a dummy example, the above code first declares two vectors. Then, a
right hand side vector is calculated as the matrix-vector product of a random vector
with the operator $A$. Finally, an initial guess is initialized with zeros.

Then, one can define a \texttt{Belos::LinearProblem} object where the
\texttt{ifpack2Preconditioner} is used for left preconditioning.
\begin{lstlisting}[language=C++]
 typedef Belos::LinearProblem<Scalar, Tpetra_MultiVector, Tpetra_Operator>
   Belos_LinearProblem;

 Teuchos::RCP<Belos_LinearProblem> problem =
   Teuchos::rcp( new Belos_LinearProblem( A, X, B ) );
 problem->setLeftPrec( mueLuPreconditioner );
 bool set = problem.setProblem();
\end{lstlisting}

Next, we set up a \belos{} solver using some basic parameters.
\begin{lstlisting}[language=C++]
 Teuchos::RCP<Teuchos::ParameterList> belosList =
   Teuchos::rcp(new Teuchos::ParameterList);
 belosList->set( "Block Size", 1 );
 belosList->set( "Maximum Iterations", 100 );
 belosList->set( "Convergence Tolerance", 1e-10 );
 belosList->set( "Output Frequency", 1 );
 belosList->set( "Verbosity", Belos::TimingDetails + Belos::FinalSummary );

 Belos::SolverFactory<Scalar, Tpetra_MultiVector, Tpetra_Operator> solverFactory;
 Teuchos::RCP<Belos::SolverManager<Scalar, Tpetra_MultiVector, Tpetra_Operator> >
   solver = solverFactory.create( "Block CG", belosList );
 solver->setProblem( problem );
\end{lstlisting}

Finally, we solve the system.
\begin{lstlisting}[language=C++]
 Belos::ReturnType ret = solver.solve();
\end{lstlisting}

It is often more convenient to specify the parameters as part of an XML-formatted options file.
Look in the subdirectory {\tt Trilinos/packages/ifpack2/test/belos} for examples of this.

This section is only meant to give a brief introduction on how to use
\ifpacktwo{} as a preconditioner within the \trilinos{} packages for iterative
solvers. There are other, more complicated, ways to use to work with
\ifpacktwo{}. For more information on these topics, the reader may refer to the
examples and tests in the \ifpacktwo{} source directory
(\texttt{Trilinos/packages/ifpack2}).


    %-----------------------------%
    \chapter{\ifpacktwo options}
    \label{sec:options}
In this section, we report the complete list of input parameters. Input
parameters are passed to \ifpacktwo in a single \parameterlist.

In some cases, the parameter types may depend on runtime template parameters.
In such cases, we will follow the conventions in Table~\ref{tab:conventions}.

\begin{table}[htbp]
  \centering
  \begin{tabular}{p{13.3cm} p{2.5cm}}
    \toprule
    \verb!MatrixType::local_ordinal_type!                                  & \verb!local_ordinal! \\
    \verb!MatrixType::global_ordinal_type!                                 & \verb!global_ordinal! \\
    \verb!MatrixType::scalar_type!                                         & \verb!scalar! \\
    \verb!MatrixType::node_type!                                           & \verb!node! \\
    \verb!Tpetra::Vector<scalar,local_ordinal,global_ordinal,node>!        & \verb!vector!\\
    \verb!Tpetra::MultiVector<scalar,local_ordinal,global_ordinal,node>!   & \verb!multi_vector!\\
    \verb!vector::mag_type!                                                & \verb!magnitude! \\
    \bottomrule
  \end{tabular}
  \caption{\label{tab:conventions}Conventions for option types that depend on templates.}
\end{table}

\noindent\textbf{Note:} if \verb!scalar! is \texttt{double}, then \verb!magnitude! is also \texttt{double}.

\section{Point relaxation}\label{s:relaxation}

\textbf{Preconditioner type:} ``RELAXATION''.

\ifpacktwo{} implements the following classical relaxation methods: Jacobi (with
optional damping), Gauss-Seidel, Successive Over-Relaxation (SOR), symmetric
version of Gauss-Seidel and SOR. \ifpacktwo{} calls both Gauss-Seidel and SOR
"Gauss-Seidel". The algorithmic details can be found in~\cite{Saad2003}.

Besides the classical relaxation methods, \ifpacktwo{} also implements $l_1$
variants of Jacobi and Gauss-Seidel methods proposed in~\cite{Baker2011}, which
lead to a better performance in parallel applications.

\noindent{\bf Note:} if a user provides a \texttt{Tpetra::BlockCrsMatrix}, the point relaxation
methods become block relaxation methods, such as block Jacobi or block
Gauss-Seidel.

The following parameters are used in the point relaxation methods:

\ccc{relaxation: type}
    {string}
    {``Jacobi''}
    {Relaxation method to use. Accepted values: ``Jacobi'',
     ``Gauss-Seidel'', ``Symmetric Gauss-Seidel''.}
\ccc{relaxation: sweeps}
    {int}
    {1}
    {Number of sweeps of the relaxation.}
\ccc{relaxation: damping factor}
    {scalar}
    {1.0}
    {The value of the damping factor $\omega$ for the relaxation.}
\ccc{relaxation: backward mode}
    {bool}
    {\false}
    {Governs whether Gauss-Seidel is done in forward-mode (\false) or
     backward-mode (\true). Only valid for ``Gauss-Seidel'' type.}
\ccc{relaxation: use l1}
    {bool}
    {\false}
    {Use the $l_1$ variant of Jacobi or Gauss-Seidel.}
\ccc{relaxation: l1 eta}
    {magnitude}
    {1.5}
    {$\eta$ parameter for $l_1$ variant of Gauss-Seidel. Only used if
     {\tt "relaxation: use l1"} is \true.}
\ccc{relaxation: zero starting solution}
    {bool}
    {\true}
    {Governs whether or not \ifpacktwo{} uses existing values in the left hand
     side vector. If true, \ifpacktwo{} fill it with zeros before applying
     relaxation sweeps which may make the first sweep more efficient.}
\ccc{relaxation: fix tiny diagonal entries}
    {bool}
    {\false}
    {If true, the compute() method will do extra work (computation only, no MPI
     communication) to fix diagonal entries. Specifically, the diagonal values
     with a magnitude smaller than the magnitude of the threshold \texttt{relaxation: min
     diagonal value} are increased to threshold for the diagonal inversion. The
     matrix is not modified, instead the updated diagonal values are stored. If the
     threshold is zero, only the diagonal entries that are exactly zero are replaced
     with a small nonzero value (machine precision).}
\ccc{relaxation: min diagonal value}
    {scalar}
    {0.0}
    {The threshold value used in {\tt "relaxation: fix tiny diagonal entries"}.
     Only used if {\tt "relaxation: fix tiny diagonal entries"} is \true.}
\ccc{relaxation: check diagonal entries}
    {bool}
    {\false}
    {If true, the \texttt{compute()} method will do extra work (both computation
     and communication) to count diagonal entries that are zero, have negative
     real part, or are small in magnitude. This information can be later shown
     in the description.}
\ccc{relaxation: local smoothing indices}
    {Teuchos::ArrayRCP<local\_ordinal>}
    {empty}
%Teuchos::ArrayRCP MatrixType::local_ordinal_type}{\texttt{Teuchos::null}}
    {A given method will only relax on the local indices listed in the
     \texttt{ArrayRCP}, in the order that they are listed. This can be used to
     reorder the relaxation, or to only relax on a subset of ids.}

\section{Block relaxation}\label{s:block_relaxation}

\textbf{Preconditioner type:} ``BLOCK\_RELAXATION''.

% \info[inline]{AP}{ILUTP cannot be constructed through {\tt Ifpack2::Factory},
% only through additive Schwarz}

\ifpacktwo{} supports block relaxation methods. Each block corresponds to a set
of degrees of freedom within a local subdomain. The blocks can be
non-overlapping or overlapping. Block relaxation can be considered as domain
decomposition within an MPI process, and should not be confused with additive
Schwarz preconditioners (see~\ref{s:schwarz}) which implement domain
decomposition across MPI processes.

There are several ways the blocks are constructed:
\begin{itemize}
  \item Linear partitioning of unknowns

    The unknowns are divided equally among a specified number of
    partitions $L$ defined by {\tt "partitioner: local parts"}. In other words,
    assuming number of unknowns $n$ is divisible by $L$, unknown $i$ will belong
    to block number $\lfloor iL/n \rfloor$.

  \item Line partitioning of unknowns

    The unknowns are grouped based on a geometric criteria which tries to
    identify degrees of freedom that form an approximate geometric line.
    Current approach uses a local line detection inspired by the work of
    Mavriplis~\cite{Mavriplis1999} for convection-diffusion. \ifpacktwo uses
    coordinate information provided by {\tt "partitioner: coordinates"} to pick
    "close" points if they are sufficiently far away from the "far" points. It
    also makes sure the line can never double back on itself.

    These "line" partitions were found to be very beneficent to problems on
    highly anisotropic geometries such as ice-sheet simulations.

  \item User partitioning of unknowns

    The unknowns are grouped according to a user provided partition. A user
    may provide a non-overlapping partition {\tt "partitioner: map"} or an
    overlapping one {\tt "partitioner: parts"}.

    A particular example of a smoother using this approach is a Vanka
    smoother~\cite{Vanka1986}, where a user may in {\tt "partition: parts"} pressure
    degrees of freedom, and request a overlap of one thus constructing Vanka
    blocks.
\end{itemize}
The original partitioning may be further modified with {\tt "partitioner: overlap"}
parameter which will use the local matrix graph to construct overlapping
partitions.

The blocks are applied in the order they were constructed. This means that in
the case of overlap the entries in the solution vector relaxed by one block may
later be overwritten by relaxing another block.

The following parameters are used in the block relaxation methods:

\cccc{relaxation: type}
    {See~\ref{s:relaxation}.}
\cccc{relaxation: sweeps}
    {See~\ref{s:relaxation}.}
\cccc{relaxation: damping factor}
    {See~\ref{s:relaxation}.}
\cccc{relaxation: zero starting solution}
    {See~\ref{s:relaxation}.}
\cccc{relaxation: backward mode}
    {See~\ref{s:relaxation}. Currently has no effect. }
\ccc{partitioner: type}
    {string}
    {``linear''}
    {The partitioner to use for defining the blocks.  This can be either
     ``linear'', ``line'' or ``user''.}
\ccc{partitioner: overlap}
    {int}
    {0}
    {The amount of overlap between partitions (0 corresponds to no overlap).
     Only valid for ``Jacobi'' relaxation.}
\ccc{partitioner: local parts}
    {int}
    {1}
    {Number of local partitions (1 corresponds to one local partition, which
     means "do not partition locally"). Only valid for ``linear'' partitioner
     type.}
\ccc{partitioner: map}
    {Teuchos::ArrayRCP<local\_ordinal>}
    {empty}
    {An array containing the partition number for each element.
     The $i$th entry in the \texttt{ArrayRCP} is the part (block) number that
     row $i$ belongs to. Use this option if the parts (blocks) do not
     overlap. Only valid for ``user'' partitioner type.}
\ccc{partitioner: parts}
    {Teuchos::Array<Teuchos::ArrayRCP\\<local\_ordinal>>}
    {empty}
    {Use this option if the parts (blocks) overlap. The $i$th entry in the
     \texttt{Array} is an \texttt{ArrayRCP} that contains all the rows in part
     (block) $i$. Only valid for ``user'' partitioner type.}
\ccc{partitioner: line detection threshold}
    {magnitude}
    {0.0}
    {Threshold used in line detection. If the distance between two connected
     points $i$ and $j$ is within the threshold times maximum distance of all
     points connected to $i$, then point $j$ is considered close enough to line
     smooth. Only valid for ``line'' partition type.}
\ccc{partitioner: PDE equations}
    {int}
    {1}
    {Number of equations per node. Only valid for ``line'' partition type.}
\ccc{partitioner: coordinates}
    {Teuchos::RCP<multi\_vector>}
    {null}
    {Coordinates of local nodes. Only valid for ``line'' partitioner type.}
\ccc{partitioner: print level}
    {bool}
    {\false}
    {If true, produce extra information about used blocks.}

\section{Chebyshev}\label{s:Chebyshev}

\textbf{Preconditioner type:} ``CHEBYSHEV''.

% Mark Hoemmen (2016/05/31):
%   The "textbook version" of Chebyshev doesn't really
%   work; we need to get rid of it.

\ifpacktwo{} implements a variant of Chebyshev iterative method following
\ifpack{}'s implementation.  \ifpack{} has a special-case modification of the
eigenvalue bounds for the case where the maximum eigenvalue estimate is close to
one. Experiments show that the \ifpack{} imitation is much less sensitive to the
eigenvalue bounds than the textbook version.

\ifpacktwo{} uses the diagonal of the matrix to precondition the linear system on the
left. Diagonal elements less than machine precision are replaced with machine
precision.

\ifpacktwo{} requires can take any matrix $A$ but can only guarantee convergence
for real valued symmetric positive definite matrices.
\iffalse
If users could provide the ellipse parameters ($d$ and $c$ in the literature,
where $d$ is the real-valued center of the ellipse, and $d-c$ and $d+c$ the two
foci), the iteration itself would work fine with nonsymmetric real-valued $A$,
as long as the eigenvalues of $A$ can be bounded in an ellipse that is entirely
to the right of the origin.
\unsure[inline]{AP}{Really unsure about Chebyshev nonsymmetric matrices. There does not
seem anything in the code to work with ellipse. I need to ask Mark Hoemmen
about this.}
\fi

The following parameters are used in the Chebyshev method:

\ccc{chebyshev: degree}
    {int}
    {1}
    {Degree of the Chebyshev polynomial, or the number of iterations. This
     overrides parameters {\tt "relaxation: sweeps"} and {\tt "smoother: sweeps"}.}
\cccc{relaxation: sweeps}
    {Same as {\tt "chebyshev: degree"}, for compatibility with \ifpack{}.}
\cccc{smoother: sweeps}
    {Same as {\tt "chebyshev: degree"}, for compatibility with \ml{}.}
\ccc{chebyshev: max eigenvalue}
    {scalar|double}
    {computed}
    {An upper bound of the matrix eigenvalues. If not provided, the value will
     be computed by power method (see parameters {\tt "eigen-analysis: type"} and
     {\tt "chebyshev: eigenvalue max iterations"}).}
\ccc{chebyshev: min eigenvalue}
    {scalar|double}
    {computed}
    {A lower bound of the matrix eigenvalues.  If not provided, \ifpacktwo{}
     will provide an estimate based on the maximum eigenvalue and the ratio.}
\ccc{chebyshev: ratio eigenvalue}
    {scalar|double}
    {30.0}
    {The ratio of the maximum and minimum estimates of the matrix
     eigenvalues.}
\cccc{smoother: Chebyshev alpha}
    {Same as {\tt "chebyshev: ratio eigenvalue"}, for compatibility with \ml{}.}
% \ccc{chebyshev: textbook algorithm}
    % {bool}
    % {\false}
    % {If true, use the textbook variant; otherwise, use the \ifpack{} variant.}
\ccc{chebyshev: compute max residual norm}
    {bool}
    {\false}
    {The \texttt{apply} call will optionally return the norm of the residual.}
\ccc{eigen-analysis: type}
    {string}
    {"power-method"}
    {The algorithm for estimating the max eigenvalue. Currently only supports
     power method ("power-method" or "power method"). The cost of the procedure is
     roughly equal to several matrix-vector multiplications.}
\ccc{chebyshev: eigenvalue max iterations}
    {int}
    {10}
    {Number of iterations to be used in calculating the estimate for the maximum
     eigenvalue, if it is not provided by the user.}
\cccc{eigen-analysis: iterations}
    {Same as {\tt "chebyshev: eigenvalue max iterations"}, for compatibility with \ml{}.}
\ccc{chebyshev: min diagonal value}
    {scalar}
    {0.0}
    {Values on the diagonal smaller than this value are increased to this value
     for the diagonal inversion.}
\ccc{chebyshev: boost factor}
    {double}
    {1.1}
    {Factor used to increase the estimate of matrix maximum eigenvalue to ensure
    the high-energy modes are not magnified by a smoother.}
\ccc{chebyshev: assume matrix does not change}
    {bool}
    {\false}
    {Whether \texttt{compute()} should assume that the matrix has not changed
     since the last call to \texttt{compute()}. If true, \texttt{compute()}
     will not recompute inverse diagonal or eigenvalue estimates.}
\ccc{chebyshev: operator inv diagonal}
    {Teuchos::RCP<const vector>|\\Teuchos::RCP<vector>|const vector*|\\vector}
    {Teuchos::null}
    {If nonnull, a deep copy of this vector will be used as the inverse
     diagonal of the matrix, instead of computing it. Expert use only.}
\ccc{chebyshev: min diagonal value}
    {scalar}
    {machine precision}
    {If any entry of the matrix diagonal is less that this in magnitude, it will
     be replaced with this value in the inverse diagonal used for left scaling.}
\cccc{chebyshev: zero starting solution}
    {See {\tt "relaxation: zero starting solution"}.}

\section{Incomplete factorizations}

\subsection{ILU($k$)}\label{s:ILU}

\textbf{Preconditioner type:} ``RILUK''.

\ifpacktwo{} implements a standard and modified (MILU) variants of the
ILU($k$) factorization~\cite{Saad2003}. In addition, it also provides an
optional \textit{a priori} modification of the diagonal entries of a matrix to
improve the stability of the factorization.

The following parameters are used in the ILU($k$) method:

\ccc{fact: iluk level-of-fill}
    {int|global\_ordinal|magnitude|double}
    {0}
    {Level-of-fill of the factorization.}
\ccc{fact: relax value}
    {magnitude|double}
    {0.0}
    {MILU diagonal compensation value. Entries dropped during factorization
     times this factor are added to diagonal entries.}
\ccc{fact: absolute threshold}
    {magnitude|double}
    {0.0}
    {Prior to the factorization, each diagonal entry is updated by adding
     this value (with the sign of the actual diagonal entry). Can be combined
     with {\tt "fact: relative threshold"}. The matrix remains unchanged.}
\ccc{fact: relative threshold}
    {magnitude|double}
    {1.0}
    {Prior to the factorization, each diagonal element is scaled by this factor
     (not including contribution specified by {\tt "fact: absolute
     threshold"}). Can be combined with {\tt "fact: absolute threshold"}.
     The matrix remains unchanged.}
% All overlap-related code was removed by M. Hoemmen in
%
% commit 162f64572fbf93e2cac73e3034d76a3db918a494
% Author: Mark Hoemmen <mhoemme@sandia.gov>
% Date:   Fri Jan 24 17:16:19 2014 -0700
%
%     Ifpack2: RILUK: Removed all overlap-related code.
%
%     Overlap never had a correct implementation in RILUK.  Furthermore,
%     AdditiveSchwarz is the proper place for overlap to be implemented, not
%     RILUK.  Ifpack2's incomplete factorizations are local (per MPI
%     process) solvers and don't need to know anything about overlap across
%     processes.  Thus, this commit removes all overlap-related code from
%     RILUK.
%
% So, older parameter "fact: iluk level-of-overlap" is no longer valid and is ignored.

\subsection{ILUT}\label{s:ILUT}

\textbf{Preconditioner type:} ``ILUT''.

\ifpacktwo{} implements a slightly modified variant of the standard ILU factorization with specified fill and
drop tolerance ILUT($p,\tau$)~\cite{Saad1994}. The modifications follow the \aztecoo implementation.
The main difference between the \ifpacktwo implementation and the algorithm in \cite{Saad1994} is the definition of
\texttt{fact: ilut level-of-fill}.

The following parameters are used in the ILUT method:

\ccc{fact: ilut level-of-fill}
    {int|magnitude|double}
    {1}
    {Maximum number of entries to keep in each row of $L$ and $U$. Each row of
     $L$ ($U$) will have at most $\lceil\frac{(\mbox{\small\tt
     level-of-fill}-1)nnz(A)}{2n}\rceil$ nonzero entries, where $nnz(A)$ is the
     number of nonzero entries in the matrix, and $n$ is the number of rows.
     ILUT always keeps the diagonal entry in the current row, regardless of the
     drop tolerance or fill level. \textbf{Note:} \textit{This is
     different from the $p$ in the classic algorithm in~\cite{Saad1994}.}}
\ccc{fact: drop tolerance}
    {magnitude|double}
    {0.0}
    {A threshold for dropping entries ($\tau$ in~\cite{Saad1994}).}
\cccc{fact: absolute threshold}
    {See~\ref{s:ILU}.}
\cccc{fact: relative threshold}
    {See~\ref{s:ILU}.}
\cccc{fact: relax value}
    {Currently has no effect. For backwards compatibility only.}

\subsection{ILUTP}\label{s:ILUTP}

\textbf{Preconditioner type:} ``AMESOS2''.

% \info[inline]{AP}{ILUTP cannot be constructed through {\tt Ifpack2::Factory},
% only through additive Schwarz}

\ifpacktwo{} implements a standard ILUTP factorization~\cite{Saad2003}. This is
done through is through the \amesostwo interface to SuperLU~\cite{Li2011}. We
reproduce the \amesostwo options here for convenience. {\em You should consider
the \href{http://trilinos.org/docs/dev/packages/amesos2/doc/html/group__amesos2__solver__parameters.html#superlu_parameters}{\amesostwo
documentation} to be the final authority.}

The following parameters are used in the ILUTP method:

\ccc{ILU\_DropTol}
    {double}
    {1e-4}
    {ILUT drop tolerance.}
\ccc{ILU\_FillFactor}
    {double}
    {10.0}
    {ILUT fill factor.}
\ccc{ILU\_Norm}
    {string}
    {``INF\_NORM''}
    {Norm to be used in factorization. Accepted values: ``ONE\_NORM'', ``TWO\_NORM'', or ``INF\_NORM''.}
\ccc{ILU\_MILU}
    {string}
    {``SILU''}
    {Type of modified ILU to use. Accepted values: ``SILU'', ``SMILU\_1'', ``SMILU\_2'', or ``SMILU\_3''.}


\section{Additive Schwarz}\label{s:schwarz}

\textbf{Preconditioner type:} ``SCHWARZ''.

\ifpacktwo{} implements additive Schwarz domain decomposition with optional
overlap. Each subdomain corresponds to exactly one MPI process in the given
matrix's MPI communication. For domain decomposition within an
MPI process see~\ref{s:block_relaxation}.

One-level overlapping domain decomposition preconditioners use local solvers of
Dirichlet type. This means that the inverse of the local matrix (possibly with
overlap) is applied to the residual to be preconditioned. The preconditioner can
be written as:
$$ P_{AS}^{-1} = \sum_{i=1}^M P_i A_i^{-1} R_i, $$
where $M$ is the number of subdomains (in this case, the number of (MPI)
processes in the computation), $R_i$ is an operator that restricts the global
vector to the vector lying on subdomain $i$, $P_i$ is the prolongator
operator, and $A_i = R_i A P_i$.

Constructing a Schwarz preconditioner requires defining two components.

{\bf Definition of the restriction and prolongation operators.}
Users may control how the data is combined with existing data by setting {\tt
"combine mode"} parameter. Table~\ref{t:combine_mode} contains a list of modes to
combine overlapped entries. The default mode is ``ZERO'' which is equivalent to
using a restricted additive Schwarz~\cite{Cai1999} method.

\begin{table}[htbp]
  \centering
  \begin{tabular}{p{3.5cm} p{12.0cm}}
    \toprule
    Combine mode name & Description \\
    \midrule
    ``ADD''           & Sum values into existing values \\
    ``ZERO''          & Replace old values with zero \\
    ``INSERT''        & Insert new values that don't currently exist \\
    ``REPLACE''       & Replace existing values with new values \\
    ``ABSMAX''        & Replace old values with maximum of magnitudes of old and new values \\
    \bottomrule
  \end{tabular}
  \caption{\label{t:combine_mode}Combine mode descriptions.}
\end{table}

{\bf Definition of a solver for subdomain linear system.}
Some preconditioners may benefit from local modifications to the subdomain
matrix. It can be filtered to eliminate singletons and/or reordered.
Reordering will often improve performance during incomplete factorization setup,
and improve the convergence. The matrix reordering algorithms specified in {\tt
"schwarz: reordering list"} are provided by \zoltantwo.  At the present time,
the only available reordering algorithm is RCM (reverse Cuthill-McKee). Other
orderings will be supported by the Zoltan2 package in the future.

To solve linear systems involving $A_i$ on each subdomain, a user can specify
the inner solver by setting {\tt "inner preconditioner name"} parameter (or any
of its aliases) which allows to use any \ifpacktwo preconditioner. These include
but are not necessarily limited to the preconditioners in
Table~\ref{t:schwarz_inner}.

\begin{table}[htbp]
  \centering
  \begin{tabular}{p{5.0cm} p{10.5cm}}
    \toprule
    Inner solver type       & Description \\
    \midrule
    ``DIAGONAL''            & Diagonal scaling \\
    ``RELAXATION''          & Point relaxation (see~\ref{s:relaxation}) \\
    ``BLOCK\_RELAXATION''   & Block relaxation (see~\ref{s:block_relaxation}) \\
    ``CHEBYSHEV''           & Chebyshev iteration (see~\ref{s:Chebyshev}) \\
    ``RILUK''               & ILU($k$) (see~\ref{s:ILU}) \\
    ``ILUT''                & ILUT (see~\ref{s:ILUT}) \\
    ``AMESOS2''             & \amesostwo's interface to sparse direct solvers \\
    ``DENSE'' or ``LAPACK'' & LAPACK's LU factorization for a dense representation of a subdomain matrix \\
    ``CUSTOM''              & User provided inner solver \\
    % ``RBILUK''
    \bottomrule
  \end{tabular}
  \caption{\label{t:schwarz_inner}Additive Schwarz solver preconditioner types.}
\end{table}

The following parameters are used in the Schwarz method:

\ccc{schwarz: inner preconditioner name}
    {string}
    {none}
    {The name of the subdomain solver.}
\cccc{inner preconditioner name}
    {Same as {\tt "schwarz: inner preconditioner name"}.}
\cccc{schwarz: subdomain solver name}
    {Same as {\tt "schwarz: inner preconditioner name"}.}
\cccc{subdomain solver name}
    {Same as {\tt "schwarz: inner preconditioner name"}.}
\ccc{schwarz: inner preconditioner parameters}
    {\parameterlist}
    {empty}
    {Parameters for the subdomain solver. If not provided, the subdomain solver
     will use its specific default parameters.}
\cccc{inner preconditioner parameters}
    {Same as {\tt "schwarz: inner preconditioner parameters"}.}
\cccc{schwarz: subdomain solver parameters}
    {Same as {\tt "schwarz: inner preconditioner parameters"}.}
\cccc{subdomain solver parameters}
    {Same as {\tt "schwarz: inner preconditioner parameters"}.}
\ccc{schwarz: combine mode}
    {string}
    {``ZERO''}
    {The rule for combining incoming data with existing data in overlap regions.
     Accepted values: see Table~\ref{t:combine_mode}.}
\ccc{schwarz: overlap level}
    {int}
    {0}
    {The level of overlap (0 corresponds to no overlap).}
\ccc{schwarz: num iterations}
    {int}
    {1}
    {Number of iterations to perform.}
\ccc{schwarz: use reordering}
    {bool}
    {\false}
    {If true, local matrix is reordered before computing subdomain solver. \trilinos must have been built with
     \zoltantwo and \xpetra enabled.}
\ccc{schwarz: reordering list}
    {\parameterlist}
    {empty}
    {Specify options for a \zoltantwo reordering algorithm to use. See {\tt
     "order\_method"}. {\em You should consider the
     \href{http://trilinos.org/docs/dev/packages/zoltan2/doc/html/z2_parameters.html}{\zoltantwo
     documentation} to be the final authority.}}
\ccc{order\_method}
    {string}
    {``rcm''}
    {Reordering algorithm. Accepted values: ``rcm'', ``minimum\_degree'',
     ``natural'', ``random'', or ``sorted\_degree''. Only used in {\tt
     "schwarz: reordering list"} sublist.}
\cccc{schwarz: zero starting solution}
    {See {\tt "relaxation: zero starting solution"}.}
\ccc{schwarz: filter singletons}
    {bool}
    {\false}
    {If true, exclude rows with just a single entry on the calling process.}
\cccc{schwarz: subdomain id}
    {Currently has no effect.}
\cccc{schwarz: compute condest}
    {Currently has no effect. For backwards compatibility only.}

\section{Hiptmair}

\ifpacktwo{} implements Hiptmair algorithm of~\cite{Hiptmair1997}. The method
operates on two spaces: a primary space and an auxiliary space. This situation
arises, for instance,  when preconditioning Maxwell's equations discretized by
edge elements. It is used in \muelu~\cite{MueLu} ``RefMaxwell''
solver~\cite{RefMaxwell}.

Hiptmair's algorithm does not use \texttt{Ifpack2::Factory} interface for
construction.  Instead, a user must explicitly call the constructor
\begin{lstlisting}[language=C++]
  Teuchos::RCP<Tpetra::CrsMatrix<> > A, Aaux, P;
  // create A, Aaux, P here ...
  Teuchos::ParameterList paramList;
  paramList.set("hiptmair: smoother type 1", "CHEBYSHEV");
  // ...
  RCP<Ifpack2::Ifpack2Preconditioner<> > ifpack2Preconditioner =
    Teuchos::rcp(new Ifpack2::Hiptmair(A, Aaux, P);
  ifpack2Preconditioner->setParameters(paramList);
\end{lstlisting}
\noindent Here, $A$ is a matrix in the primary space, $Aaux$ is a matrix in
auxiliary space, and $P$ is a prolongator/restrictor between the two spaces.

The following parameters are used in the Hiptmair method:

\ccc{hiptmair: smoother type 1}
    {string}
    {"CHEBYSHEV"}
    {Smoother type for smoothing the primary space.}
\ccc{hiptmair: smoother list 1}
    {\parameterlist}
    {empty}
    {Smoother parameters for smoothing the primary space.}
\ccc{hiptmair: smoother type 2}
    {string}
    {"CHEBYSHEV"}
    {Smoother type for smoothing the auxiliary space.}
\ccc{hiptmair: smoother list 2}
    {\parameterlist}
    {empty}
    {Smoother parameters for smoothing the auxiliary space.}
\ccc{hiptmair: pre or post}
    {string}
    {``both''}
    {\ifpacktwo{} always relaxes on the auxiliary space. ``pre'' (``post'') means
     that it relaxes on the primary space before (after) the relaxation on the
     auxiliary space. ``both'' means that we do both ``pre'' and ``post''.}
\cccc{hiptmair: zero starting solution}
    {See {\tt "relaxation: zero starting solution"}.}


    %-----------------------------%
    % \chapter{Performance}
    % In practice, it can be very challenging to find an appropriate set of multigrid
parameters for a specific problem, especially if few details are known about the
underlying linear system. In this Chapter, we provide some advice for improving
multigrid performance.

\begin{mycomment}
For optimizing multigrid parameters, it is highly recommended to set the
verbosity to \verb|high| or \verb|extreme| for \muelu{} to output more
information (e.g., for the effect of the chosen parameters to the aggregation
and coarsening process).
\end{mycomment}

Some general advice:
\begin{itemize}
  \item
    Choose appropriate iterative linear solver (e.g., GMRES for non-symmetric problems).

  \item
    Start with the recommended settings for particular problem types. See
    Table~\ref{t:problem_types}.

  \item
    Choose reasonable basic multigrid parameters
    (see~\S\ref{sec:options_general}), including maximum number of multigrid
    levels (\texttt{max levels}) and maximum allowed coarse size of the problem
    (\texttt{coarse: max size}). Take fine level problem size and sparsity
    pattern into account for a reasonable choice of these parameters.

  \item
    Select an appropriate transfer operator strategy
    (see~\S\ref{sec:options_mg}). For symmetric problems, you should start with smoothed
    aggregation multigrid. For non-symmetric problems, a Petrov-Galerkin smoothed
    aggregation method is a good starting point, though smoothed aggregation may
    also perform well.

  \item
    Enable implicit restrictor construction (\texttt{transpose:} \texttt{use implicit}) for symmetric
    problems.

  \item
    Find good level smoothers (see~\S\ref{sec:options_smoothing}). If a problem
    is symmetric positive definite, choose a smoother with a matrix-vector
    computational kernel, such as the Chebyshev polynomial smoother. If you are
    using relaxation smoothers, we recommend starting with optimizing the
    damping parameter. Once you have found a good damping parameter for your
    problem, you can increase the number of smoothing iterations.

  \item
    Adjust aggregation parameters if you experience bad coarsening ratios
    (see~\S\ref{sec:options_aggregation}). Particularly, try adjusting the
    minimum (\texttt{aggregation:} \texttt{min agg size}) and maximum
    (\texttt{aggregation:} \texttt{max agg size}) aggregation parameters. For a
    2D (3D) isotropic problem on a regular mesh, the aggregate size should be
    about 9 (27) nodes per aggregate.

  \item
    Replace a direct solver with an iterative method (\texttt{coarse: type}) if
    your coarse level solution becomes too expensive (see~\S\ref{sec:options_smoothing}).
\end{itemize}

Some advice for parallel runs include:
\begin{enumerate}
  \item
    Enable matrix rebalancing when running in parallel (\texttt{repartition:}
    \texttt{enable}).

  \item
    Use smoothers invariant to the number of processors, such as
    polynomial smoothing, if possible.

  \item
    Use \texttt{uncoupled} aggregation instead of \texttt{coupled}, as later
    requires significantly more communication.

  \item
    Adjust rebalancing parameters (see~\S\ref{sec:options_rebalancing}). Try
    choosing rebalancing parameters so that you end up with one processor on the
    coarsest level for the direct solver (this avoids additional communication).

  \item
    Enable implicit rebalancing of prolongators and restrictors
    (\texttt{repartition: rebalance P and R}).
\end{enumerate}


    %\nocite{*}

    % ---------------------------------------------------------------------- %
    % References
    %
    \clearpage
    % If hyperref is included, then \phantomsection is already defined.
    % If not, we need to define it.
    \providecommand*{\phantomsection}{}
    \phantomsection
    \addcontentsline{toc}{chapter}{References}
    \bibliographystyle{plain}
    \bibliography{ifpack2guide}


    % ---------------------------------------------------------------------- %
    %
    \appendix
    \input{appendix}
    %\chapter{Historical Perspective}
	%\input{CommonHistory}


    %\chapter{Some Other Appendix}
	%\input{CommonAppendix}

    % \printindex

    %
% This is an example of how to create the distribution page. Some
% distributions are required by Sandia; e.g. the housekeeping copies.
% Depending on the type of report; e.g. CRADA, Patent Caution, etc.
% additional distribution lines may have to be added. See the
% "Guide for Preparing SAND Reports"
%
% SANDdistribution takes CA or NM as an optional argument. If given,
% the approrpiate housekeeping copies are inserted autmatically.
% Inside the SANDdistribution environment, several commands can be used
% insert the distributions for CRADA, LDRD, etc. See example below.
%
% You can leave the CA or NM option off and not use any of the SANDdist*
% commands. This will allow you to create a distribution list manually.
%
\begin{SANDdistribution}[NM]
    % Housekeeping copies necessary for every unclassified report:
    % \SANDdistCRADA	% If this report is about CRADA work
    % \SANDdistPatent	% If this report has a Patent Caution or Patent Interest
    % \SANDdistLDRD	% If this report is about LDRD work

    % Some external Addresses

    % \SANDdistExternal{3}{Some Address\\ and street\\City, State}
    % \SANDdistExternal{12}{Another Address\\ On a street\\City, State\\U.S.A.}
    %\bigskip

    % The following MUST BE between the external and internal distributions!
    % \SANDdistClassified % If this report is classified

    % Internal Addresses
    \SANDdistInternal{1}{9018}{Central Technical Files}{8945-1}
    \SANDdistInternal{2}{0899}{Technical Library}{9610}
    \SANDdistInternal{2}{0612}{Review \& Approval Desk}{4916}
    \SANDdistInternal{1}{1320}{Eric C. Cyr}{1442}
    \SANDdistInternal{1}{1320}{Curtis C. Ober}{1446}
    \SANDdistInternal{1}{1318}{Roger P. Pawlowski}{1446}
    
    % Example of a mail channel use (instead of a mail stop)
    % \SANDdistInternalM{1}{M9999}{Someone}{01234}

\end{SANDdistribution}


\end{document}
